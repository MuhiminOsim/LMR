\subsubsection{Dirichlet inversion}
Define the Dirichlet convolution $f*g(n)$ as:

$$f*g(n)=\sum^n_{d=1}[d|n]f(n)g(\frac{n}{d})$$

Assume we are going to calculate some function $S(n)=\sum^n_{i=1}f(i)$,
where $f(n)$ is a multiplicative function.
Say we find some $g(n)$ that is simple to calculate,
and $\sum^n_{i=1}f*g(i)$ can be figured out in $O(1)$ complexity.
Then we have

\begin{equation*}
\begin{split}
\sum^n_{i=1}f*g(i)	&=\sum^n_{i=1}\sum_d[d|i]g(\frac{i}{d})f(d)\\
					&=\sum^n_{\frac{i}{d}=1}\sum^{\floor*{\frac{n}{\frac{i}{d}}}}_{d=1}g(\frac{i}{d})f(d)\\
					&=\sum^n_{i=1}\sum^{\floor*{\frac{n}{i}}}_{d=1}g(i)f(d)\\
					&=g(1)S(n)+\sum^n_{i=2}g(i)S(\floor*{\frac{n}{i}})\\
S(n)				&=\frac{\sum^n_{i=1}f*g(i)-\sum^n_{i=2}g(i)S(\floor*{\frac{n}{i}})}{g(1)}\\
\end{split}
\end{equation*}

It can be proven that $\floor*{\frac{n}{i}}$ has at most $O(\sqrt{n})$ possible values.
Therefore, the calculation of $S(n)$ can be reduced to $O(\sqrt{n})$ calculations of $S(\floor*{\frac{n}{i}})$.
By applying the master theorem, it can be shown that the complexity of such method is $O(n^{\frac{3}{4}})$.

Moreover, since $f(n)$ is multiplicative, we can process the first $n^{\frac{2}{3}}$ elements via linear sieve,
and for the rest of the elements, we apply the method shown above. The complexity can thus be enhanced to $O(n^{\frac{2}{3}})$.

For the prefix sum of Euler's function $S(n)=\sum^n_{i=1}\varphi(i)$, notice that $\sum_{d|n}\varphi(d)=n$.
Hence $\varphi*I=id$. ($I(n)=1,id(n)=n$)
Now let $g(n)=I(n)$, and we have $S(n)=\sum^n_{i=1}i-\sum^n_{i=2}S(\floor*{\frac{n}{i}})$.

For the prefix sum of Mobius function $S(n)=\sum^n_{i=1}\mu(i)$, notice that $\mu*I=(n)\{[n=1]\}$.
Hence $S(n)=1-\sum^n_{i=2}S(\floor*{\frac{n}{i}})$.

Some other convolutions include $(p^k)\{1-p\}*id=I$, $(p^k)\{p^k-p^{k+1}\}*id^2=id$ and $(p^k)\{p^{2k}-p^{2k-2}\}*I=id^2$.

Usage:
\begin{enumerate}
	\item \texttt{CUBEN} should be $N^{\frac{1}{3}}$.
	\item Pass \texttt{p\_f} that returns the prefix sum of $f(x)(1\le x<th)$.
	\item Pass \texttt{p\_g} that returns the prefix sum of $g(x)(0\le x\le N)$.
	\item Pass \texttt{p\_c} that returns the prefix sum of $f*g(x)(0\le x\le N)$.
	\item Pass \texttt{th} as the thereshold, which generally should be $N^{\frac{2}{3}}$.
	\item Pass \texttt{mod} as the module number, \texttt{inv} as the inverse of $g(1)$ regarding \texttt{mod}.
	\item \textbf{Remember that $x$ in \texttt{p\_g(x)} and \texttt{p\_c(x)} may be larger than \texttt{mod}!}
	\item Run \texttt{init(n)} first.
	\item Use \texttt{ans(x)} to fetch answer for $\frac{n}{x}$.
\end{enumerate}

\lstinputlisting{src/mathematics/computation/dirichlet-convolution.cpp}
