The enumeration theorem employs a multivariate generating function called the cycle index:
$$Z_{G}(t_{1},t_{2},\ldots ,t_{n})={\frac {1}{|G|}}\sum _{g\in G}t_{1}^{j_{1}(g)}t_{2}^{j_{2}(g)}\cdots t_{n}^{j_{n}(g)}\,,$$
where $n$ is the number of elements of $X$ and $j_k(g)$ is the number of $k$-cycles of the group element $g$ as a permutation of $X$.

The theorem states that the generating function F of the number of colored arrangements by weight is given by:
$$F(t)=Z_{G}(f(t),f(t^{2}),f(t^{3}),\ldots ,f(t^{n}))\,,$$
or in the multivariate case:
$$F(t_{1},\ldots )=Z_{G}(f(t_{1},\ldots ),f(t_{1}^{2},\ldots ),f(t_{1}^{3},\ldots ),\ldots ,f(t_{1}^{n},\ldots ))\,.$$

For instance, when seperating the graphs with the number of edges, we let $f(t)=1+t$, and examine the coefficient of $t^i$ for a graph with $i$ edges, and when seperating the necklaces with the number of beads with three different colors, we let $f(x,y,z)=x+y+z$, and examine the coefficient of $x^iy^jz^k$.
